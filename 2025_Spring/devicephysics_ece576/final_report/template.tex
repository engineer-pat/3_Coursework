% *** Authors should verify (and, if needed, correct) their LaTeX system  ***
% *** with the testflow diagnostic prior to trusting their LaTeX platform ***
% *** with production work. IEEE's font choices can trigger bugs that do  ***
% *** not appear when using other class files.                            ***
% The testflow support page is at:
% http://www.michaelshell.org/tex/testflow/


%%*************************************************************************
%% Legal Notice:
%% This code is offered as-is without any warranty either expressed or
%% implied; without even the implied warranty of MERCHANTABILITY or
%% FITNESS FOR A PARTICULAR PURPOSE!
%% User assumes all risk.
%% In no event shall IEEE or any contributor to this code be liable for
%% any damages or losses, including, but not limited to, incidental,
%% consequential, or any other damages, resulting from the use or misuse
%% of any information contained here.
%%
%% All comments are the opinions of their respective authors and are not
%% necessarily endorsed by the IEEE.
%%
%% This work is distributed under the LaTeX Project Public License (LPPL)
%% ( http://www.latex-project.org/ ) version 1.3, and may be freely used,
%% distributed and modified. A copy of the LPPL, version 1.3, is included
%% in the base LaTeX documentation of all distributions of LaTeX released
%% 2003/12/01 or later.
%% Retain all contribution notices and credits.
%% ** Modified files should be clearly indicated as such, including  **
%% ** renaming them and changing author support contact information. **
%%
%% File list of work: IEEEtran.cls, New_IEEEtran_how-to.pdf, bare_jrnl_new_sample4.tex,
%%*************************************************************************
\PassOptionsToPackage{unicode}{hyperref}
\PassOptionsToPackage{hyphens}{url}
\PassOptionsToPackage{dvipsnames,svgnames,x11names}{xcolor}
% Note that the a4paper option is mainly intended so that authors in
% countries using A4 can easily print to A4 and see how their papers will
% look in print - the typesetting of the document will not typically be
% affected with changes in paper size (but the bottom and side margins will).
% Use the testflow package mentioned above to verify correct handling of
% both paper sizes by the user's LaTeX system.
%
% Also note that the "draftcls" or "draftclsnofoot", not "draft", option
% should be used if it is desired that the figures are to be displayed in
% draft mode.
%
\documentclass[
  journal,
]{IEEEtran}%
% If IEEEtran.cls has not been installed into the LaTeX system files,
% manually specify the path to it like:
% \documentclass[journal]{../sty/IEEEtran}
\usepackage[cmex10]{amsmath}
\usepackage{amssymb}
\usepackage{iftex}
\ifPDFTeX
  \usepackage[T1]{fontenc}
  \usepackage[utf8]{inputenc}
  \usepackage{textcomp} % provide euro and other symbols
\else % if luatex or xetex
  \usepackage{unicode-math} % this also loads fontspec
  \defaultfontfeatures{Scale=MatchLowercase}
  \defaultfontfeatures[\rmfamily]{Ligatures=TeX,Scale=1}
\fi
%\usepackage{lmodern}
\ifPDFTeX\else
\fi
% Use upquote if available, for straight quotes in verbatim environments
\IfFileExists{upquote.sty}{\usepackage{upquote}}{}
\IfFileExists{microtype.sty}{% use microtype if available
  \usepackage[]{microtype}
  \UseMicrotypeSet[protrusion]{basicmath} % disable protrusion for tt fonts
}{}
\makeatletter
\parindent    1.0em
\ifCLASSOPTIONcompsoc
  \parindent    1.5em
\fi
\makeatother
\usepackage{xcolor}
\setlength{\emergencystretch}{3em} % prevent overfull lines

\setcounter{secnumdepth}{5}
% Make \paragraph and \subparagraph free-standing
\ifx\paragraph\undefined\else
  \let\oldparagraph\paragraph
  \renewcommand{\paragraph}[1]{\oldparagraph{#1}\mbox{}}
\fi
\ifx\subparagraph\undefined\else
  \let\oldsubparagraph\subparagraph
  \renewcommand{\subparagraph}[1]{\oldsubparagraph{#1}\mbox{}}
\fi


\providecommand{\tightlist}{%
  \setlength{\itemsep}{0pt}\setlength{\parskip}{0pt}}\usepackage{longtable,booktabs,array}
\usepackage{calc} % for calculating minipage widths
% Correct order of tables after \paragraph or \subparagraph
\usepackage{etoolbox}
\makeatletter
\patchcmd\longtable{\par}{\if@noskipsec\mbox{}\fi\par}{}{}
\makeatother
% Allow footnotes in longtable head/foot
\IfFileExists{footnotehyper.sty}{\usepackage{footnotehyper}}{\usepackage{footnote}}
\makesavenoteenv{longtable}
\usepackage{graphicx}
\makeatletter
\def\maxwidth{\ifdim\Gin@nat@width>\linewidth\linewidth\else\Gin@nat@width\fi}
\def\maxheight{\ifdim\Gin@nat@height>\textheight\textheight\else\Gin@nat@height\fi}
\makeatother
% Scale images if necessary, so that they will not overflow the page
% margins by default, and it is still possible to overwrite the defaults
% using explicit options in \includegraphics[width, height, ...]{}
\setkeys{Gin}{width=\maxwidth,height=\maxheight,keepaspectratio}
% Set default figure placement to htbp
\makeatletter
\def\fps@figure{htbp}
\makeatother

\usepackage{amsmath}
\usepackage{mathtools}
\usepackage{amsmath}
\usepackage{mathtools}
\usepackage{physics}
\usepackage[version=3]{mhchem}
\usepackage{orcidlink}
\usepackage{float}
\floatplacement{table}{htb}
\makeatletter
\@ifpackageloaded{caption}{}{\usepackage{caption}}
\AtBeginDocument{%
\ifdefined\contentsname
  \renewcommand*\contentsname{Table of contents}
\else
  \newcommand\contentsname{Table of contents}
\fi
\ifdefined\listfigurename
  \renewcommand*\listfigurename{List of Figures}
\else
  \newcommand\listfigurename{List of Figures}
\fi
\ifdefined\listtablename
  \renewcommand*\listtablename{List of Tables}
\else
  \newcommand\listtablename{List of Tables}
\fi
\ifdefined\figurename
  \renewcommand*\figurename{Fig.}
\else
  \newcommand\figurename{Fig.}
\fi
\ifdefined\tablename
  \renewcommand*\tablename{Table}
\else
  \newcommand\tablename{Table}
\fi
}
\@ifpackageloaded{float}{}{\usepackage{float}}
\floatstyle{ruled}
\@ifundefined{c@chapter}{\newfloat{codelisting}{h}{lop}}{\newfloat{codelisting}{h}{lop}[chapter]}
\floatname{codelisting}{Listing}
\newcommand*\listoflistings{\listof{codelisting}{List of Listings}}
\makeatother
\makeatletter
\makeatother
\makeatletter
\@ifpackageloaded{caption}{}{\usepackage{caption}}
\@ifpackageloaded{subcaption}{}{\usepackage{subcaption}}
\makeatother
\usepackage[skip=2pt,font=footnotesize]{caption}
%\captionsetup{format=myformat}
\ifLuaTeX
  \usepackage{selnolig}  % disable illegal ligatures
\fi
\IfFileExists{bookmark.sty}{\usepackage{bookmark}}{\usepackage{hyperref}}
\IfFileExists{xurl.sty}{\usepackage{xurl}}{} % add URL line breaks if available
\urlstyle{same} % disable monospaced font for URLs
\hypersetup{
  pdftitle={Evolution of Charge Carriers Through Time and Space},
  pdfauthor={Patrick Finnerty; Trinity Watson; Ralph Mora},
  colorlinks=true,
  linkcolor={blue},
  filecolor={Maroon},
  citecolor={Blue},
  urlcolor={Blue},
  pdfcreator={LaTeX via pandoc}}

% *** Do not adjust lengths that control margins, column widths, etc. ***
% *** Do not use packages that alter fonts (such as pslatex).         ***
% There should be no need to do such things with IEEEtran.cls V1.6 and later.
% (Unless specifically asked to do so by the journal or conference you plan
% to submit to, of course. )


% correct bad hyphenation here
\hyphenation{op-tical net-works semi-conduc-tor}

%
% paper title
% can use linebreaks \\ within to get better formatting as desired
% Do not put math or special symbols in the title.
% paper title
% can use linebreaks \\ within to get better formatting as desired
% Do not put math or special symbols in the title.
\title{Evolution of Charge Carriers Through Time and Space}

\author{
Patrick Finnerty\orcidlink{0009-0003-4645-6732},~Trinity Watson
and~Ralph Mora%
\thanks{Patrick Finnerty is with Electrical and Computer
Engineering, University of New Mexico, Albuquerque, NM, 87106 United
States of America%
}
%by-author.affiliations
\thanks{Corresponding author: patf@unm.edu}
\thanks{Trinity Watson is with Electrical and Computer and
Engineering, University of New Mexico, Albuquerque, NM, 87106 United
States of America%
}
%by-author.affiliations
\thanks{Ralph Mora is with Electrical and Computer
Engineering, University of New Mexico, Albuquerque, NM, 87106 United
States of America%
}
%by-author.affiliations
}
\begin{document}

% The paper headers
\markboth{Sharma Journal of Nature Science IEEE, May 2025}{}

% use for special paper notices

% make the title area
\maketitle

% As a general rule, do not put math, special symbols or citations
% in the abstract or keywords.
\begin{abstract}
The physics underlying semiconductor devices must be understood to
effectively design new devices and experiments. While equations can be
provided to model phenomena such as propagation of carriers through a
channel or device, or the evolution of current in the device with time,
conducting the derivation of these equations from parent equations and
phenomena may be highly informative. In this report, we present the
derivation of equations for carrier concentrations as a function of time
and space within a device; these equations are then used to model the
behavior of charge carriers and the resulting current within a device.
An app has been developed that allows for visualization and interaction
with these equations. The app is available at
\url{https://pat-trinity-ralph-carrier-propagation-devicephysics-unm.streamlit.app/}.
\end{abstract}
% Note that keywords are not normally used for peerreview papers.

% For peer review papers, you can put extra information on the cover
% page as needed:
% \ifCLASSOPTIONpeerreview
% \begin{center} \bfseries EDICS Category: 3-BBND \end{center}
% \fi
%
% For peerreview papers, this IEEEtran command inserts a page break and
% creates the second title. It will be ignored for other modes.
% \IEEEpeerreviewmaketitle


\section{Introduction}\label{sec-intro}

\IEEEPARstart{T}{he} current passing through a device, usually within a
channel, controlled by a gate voltage within a\\
MOSFET (Metal-Oxide-Semiconductor Field-Effect Transistor) is a critical
parameter. The current is influenced by many factors, but principally
arises from the movement of charge carriers, electrons and holes.
Understanding and modeling this movement of carriers and thereby the
current is necessary to informatively design a device. We therefore set
out to derive the relevant equations.

\section{Derivation of Carrier Concentration and
Current}\label{sec-derivation}

\subsection{Green's Functions and Fourier
Transform}\label{greens-functions-and-fourier-transform}

We start with Green's functions for the partial derivatives of electrons
and holes as functions of space and time, \(n(x,t)\) and \(p(x,t)\),
respectively:

For electrons:

\[
\begin{multlined}
\frac{\partial^2G_n(x,t)}{\partial x^{2}}
- \Bigl(\frac{\nu_n}{D_n}\Bigr)\frac{\partial G_n(x,t)}{\partial x}
\\
+ \frac{1}{D_n}\frac{\partial G_n(x,t)}{\partial t}
+ \frac{1}{\tau_n D_n}\,G_n(x,t)
= \frac{1}{D_n}\,\delta(x - x_0)\,\delta(t - t_0)
\end{multlined}
\]

For holes: \[
\begin{multlined}
\frac{\partial^2G_p(x,t)}{\partial x^{2}}
- \Bigl(\frac{\nu_p}{D_p}\Bigr)\frac{\partial G_p(x,t)}{\partial x}
\\
+ \frac{1}{D_p}\frac{\partial G_p(x,t)}{\partial t}
+ \frac{1}{\tau_p D_p}\,G_p(x,t)
= \frac{1}{D_p}\,\delta(x - x_0)\,\delta(t - t_0)
\end{multlined}
\]

The generation terms, \(G_n\) and \(G_p\), for electrons and holes
respectively: \[
G_n = G_p = \left( \frac{P_0 N}{h \nu} \right)
e^{-\frac{(x - x_0)^2}{\delta(x - x_0)}}
e^{-\frac{(t - t_0)^2}{\delta(t - t_0)}}
\]

Were input to the above Green's equations, resulting in the equations
below. For electrons: \[
\begin{multlined}
\frac{\partial\Bigl(\tfrac{P_0 N}{h\nu}
  e^{-\frac{(x - x_0)^2}{\delta(x - x_0)}}
  e^{-\frac{(t - t_0)^2}{\delta(t - t_0)}}\Bigr)}{\partial x^2}
- \Bigl(\tfrac{\nu_n}{D_n}\Bigr)\,
  \frac{\partial\Bigl(\tfrac{P_0 N}{h\nu}
  e^{-\frac{(x - x_0)^2}{\delta(x - x_0)}}
  e^{-\frac{(t - t_0)^2}{\delta(t - t_0)}}\Bigr)}{\partial x}
\\
+ \frac{1}{D_n}\,
  \frac{\partial\Bigl(\tfrac{P_0 N}{h\nu}
  e^{-\frac{(x - x_0)^2}{\delta(x - x_0)}}
  e^{-\frac{(t - t_0)^2}{\delta(t - t_0)}}\Bigr)}{\partial t}
+ \frac{1}{\tau_n D_n}\,
  \Bigl(\tfrac{P_0 N}{h\nu}
  e^{-\frac{(x - x_0)^2}{\delta(x - x_0)}}
  e^{-\frac{(t - t_0)^2}{\delta(t - t_0)}}\Bigr)
\\
= \frac{1}{D_n}\,\delta(x - x_0)\,\delta(t - t_0)
\end{multlined}
\]

For holes: \[
\begin{multlined}
\frac{\partial^2\Bigl(\frac{P_0 N}{h\nu}
\,e^{-\frac{(x - x_0)^2}{\delta(x - x_0)}}
\,e^{-\frac{(t - t_0)^2}{\delta(t - t_0)}}\Bigr)}%
{\partial x^2}
- \Bigl(\tfrac{\nu_{n}}{D_{n}}\Bigr)\,
  \frac{\partial\Bigl(\frac{P_0 N}{h\nu}
  \,e^{-\frac{(x - x_0)^2}{\delta(x - x_0)}}
  \,e^{-\frac{(t - t_0)^2}{\delta(t - t_0)}}\Bigr)}%
  {\partial x}
\\
+ \frac{1}{D_{n}}\,
  \frac{\partial\Bigl(\frac{P_0 N}{h\nu}
  \,e^{-\frac{(x - x_0)^2}{\delta(x - x_0)}}
  \,e^{-\frac{(t - t_0)^2}{\delta(t - t_0)}}\Bigr)}%
  {\partial t}
\\
+ \frac{1}{\tau_{n}D_{n}}
  \Bigl(\frac{P_0 N}{h\nu}
  \,e^{-\frac{(x - x_0)^2}{\delta(x - x_0)}}
  \,e^{-\frac{(t - t_0)^2}{\delta(t - t_0)}}\Bigr)
\\
= \frac{1}{D_{n}}\;\delta(x - x_{0})\,\delta(t - t_{0})
\end{multlined}
\]

We can define constants: \[
A = e^{-\frac{(t - t_0)^2}{\delta(t - t_0)}}\\
B =   e^{-\frac{(x - x_0)^2}{\delta(x - x_0)}}\\
C = {\frac{P_0 N}{h \nu}}
\]

Rewriting the equation, initially focusing on the electron equation: \[
\begin{multlined}
AC \frac{\partial^(
e^{-\frac{(x - x_0)^2}{\delta(x - x_0)}}
)}{\partial x^{2}}-\left( \frac{\nu_{n}}{D_{n}}AC\right)\frac{\partial(
e^{-\frac{(x - x_0)^2}{\delta(x - x_0)}}
)}{\partial x}
\\
+
\frac{1}{D_{n}} AC\frac{\partial (
e^{-\frac{(x - x_0)^2}{\delta(x - x_0)}}
)}{\partial t}
\\
+
\frac{1}{\tau_{n}D_{n}}AC
e^{-\frac{(x - x_0)^2}{\delta(x - x_0)}}
)
\\
=\frac{1}{Dn}\delta(x-x_{0})\delta(t-t_{0})
\end{multlined}
\]

Breaking the equation into portions, we take advantage of the linearity
property of the Fourier transform.

\subsection{Fourier Transform}\label{fourier-transform}

\[
\mathcal{F} \left[ C \cdot A \cdot \frac{\partial^2 B}{\partial x^2} \right]
\]

We can then bring out the constants A and C, and take the fourier
transform of B using the law of exponents.

\[
C \cdot A \cdot \mathcal{F} \left[ \frac{\partial^2}{\partial x^2}
\left( e^{-\frac{2(x - x_0)}{\delta(x - x_0)}} \right) \right]
\]

Defining a new constant, expanding the exponent and then taking the
Fourier transform of the first section yields:

\[
\begin{multlined}
d = {\frac{2}{\delta(x-x_0)}}
\\
C \cdot A \cdot e^{dx_0} \cdot \mathcal{F} \left[ \frac{\partial^2}{\partial x^2}
\left( e^{-dx} \right) \right]
\\
C \cdot A \cdot e^{dx_0} \cdot [(jw)^2 \cdot \frac{2d}{d^2+\omega^2}]
\end{multlined}
\]

Continuing the Fourier for the following sections and bringing them all
together:

\[
\begin{multlined}
C \cdot \left( \frac{2d}{d^2 + \omega^2} \right) \cdot 
\\
A \cdot
\left(
e^{d x_0} (j \omega)^2
- \frac{v_n}{D_n} (j \omega)
+ \frac{1}{D_n} \frac{\partial}{\partial t}
+ \frac{1}{\tau_n D_n}
\right)
\\
= \frac{1}{D_n} e^{-j \omega x_0} \delta(t - t_0)
\end{multlined}
\]

\subsection{Laplace Transform}\label{laplace-transform}

Once the Fourier Transform was obtained for both equations, the Laplace
Transform was taken to get \(G_n(k,s)\) and \(G_p(k,s)\).

\[
G_n(j \omega, s) = \frac{
\frac{1}{D_n*\sqrt{2\pi}} e^{-j \omega x_0}}{
 \left(e^{d x_0} (j \omega)^2
- \frac{v_n}{D_n} (j \omega)
+ \frac{s}{D_n}
+ \frac{1}{\tau_n D_n}
\right)}
\]

\[
G_p(j \omega, s) = \frac{
\frac{1}{D_p} e^{-j \omega x_0}}{
\left(e^{d x_0} (j \omega)^2
- \frac{v_p}{D_p} (j \omega)
+ \frac{s}{D_p}
+ \frac{1}{\tau_p D_p}
\right)}
\]

The following expressions can be simplified by multiplying the
denominator by \(D_n\) and \(D_p\) respectivly as well as
\(\frac{1}{\sqrt{2\pi}}\)

\[
G_n(k, s) = \frac{1}{\sqrt{2\pi}} * \frac{
e^{-j \omega x_0}}{
\cdot \left(D_n(k)^2
- v_n (k)
+ s
+ \frac{1}{\tau_n}
\right)}
\]

\[
G_p(k, s) = \frac{1}{\sqrt{2\pi}} * \frac{
e^{-j \omega x_0}}{
\cdot \left(D_p(k)^2
- v_n (k)
+ s
+ \frac{1}{\tau_n}
\right)}
\]

\subsection{Apply Inverse Laplace
Transform}\label{apply-inverse-laplace-transform}

After applying the Fourier and Laplace transforms, we now have the
following expressions: \[
% \begin{align}
G_n(k,s) = \frac{1}{\sqrt{2\pi}}\;\frac{e^{ikx_0}}{D_n k^2 - i k \nu_n + \frac{1}{\tau_n} + s} \\[1ex]
\] \[
G_p(k,s) = \frac{1}{\sqrt{2\pi}}\;\frac{e^{ikx_0}}{D_p k^2 + i k \nu_p + \frac{1}{\tau_p} + s}
% \end{align}
\]

These equations are valid but we desire to understand and model the
charge carrier and current behavior within the real time and space
domains. Therefore, we first convert back to the time domain by applying
the inverse Laplace transform.

Define: \[
a \;=\; D_p\,k^2 \;+\; i\,k\,V_p \;+\; \frac1{\tau_p}\,
\]

Since,

\[
\mathcal{L}^{-1}\Bigl\{\frac1{s+a}\Bigr\}(t)=e^{-a t}\,
\]

The inverse Laplace transform of \(G_p(k,s)\) is given by:

\[
\begin{multlined}
\mathcal{L}^{-1}\{G_p(k,s)\}(t)
=\frac{1}{\sqrt{2\pi}}\;e^{ikx_0}\;\mathcal{L}^{-1}\Bigl\{\frac1{s+a}\Bigr\}(t)
\\
=\frac{1}{\sqrt{2\pi}}\,e^{ikx_0}\,e^{-a\,t}\,
\end{multlined}
\]

\subsection{Inverse Fourier Transform}\label{inverse-fourier-transform}

And now to convert back to space domain, we apply the inverse Fourier
transform. We begin by defining \(\alpha(t)\) to allow us to match the
form of the inverse Fourier transform:

Define: \[
\alpha(t) = \frac{1}{2\pi} \frac{e^{-t/\tau_p}}{\sqrt{2\pi}}
\]

Then, \[
\begin{multlined}
g(x,t)
= \alpha(t)
  \int_{-\infty}^{\infty}
    \exp\bigl[i k x_0 - D_p\,k^2\,t - i\,k\,\nu_p\,t\bigr]
    e^{i k x}\,dk
\\
= \alpha(t)
  \int_{-\infty}^{\infty}
    \exp\bigl[-D_p\,t\,k^2 + k\,i(\nu_p\,t - x - x_0)\bigr]
    \,dk
\end{multlined}
\]

To complete the square in the exponent, define the following constants
\textbf{which are not the same as the previous definitions}: \[
A = D_p\,t
\]

\[
B = i(\nu_p\,t - x - x_0)
\]

\[
C = 0
\]

so that\\
\[
-D_p\,t\,k^2 + k\,i(\nu_p\,t - x - x_0)
= -\bigl(A\,k^2 + B\,k + C\bigr)
\]

Plugging into our expression above, we achieve:

\[
g(x,t)
= \alpha(t)\,
  \int_{-\infty}^{\infty}
    \exp\!\bigl[-\bigl(A\,k^2 + B\,k\bigr)\bigr]\,
  dk
\]

Where,\\
\[
\alpha(t) = \frac{e^{-t/\tau_p}}{\sqrt{2\pi}},\quad
A = D_p\,t,\quad
B = i(\nu_p\,t - x - x_0)
\]

The definition of ``completing the square'', or standard
Gaussian‐integral result is:

\[
\int_{-\infty}^{\infty}
e^{-\bigl(a x^2 + b x + c\bigr)}\,dx
\;=\;
\sqrt{\frac{\pi}{a}}\;
\exp\!\Bigl(\frac{b^2}{4a}-c\Bigr)
\quad 
% \Re(a)>0
\]

Plugging in all expressions for \(A B, C\) and \(\alpha(t)\) and
expanding results in:

\[
G_n(x,t)
= \frac{1}{2\pi\sqrt{2\,D_n\,t}}
  \exp\!\Bigl(-\frac{(x - x_0 - \nu_n\,t)^{2} + \tfrac{t}{\tau_n}}{4\,D_n\,t}\Bigr),
\]

\[
G_p(x,t)
= \frac{1}{2\pi\sqrt{2\,D_p\,t}}
  \exp\!\Bigl(-\frac{(x - x_0 + \nu_p\,t)^{2} + \tfrac{t}{\tau_p}}{4\,D_p\,t}\Bigr)
\]

\subsection*{Final Integration}\label{final-integration}
\addcontentsline{toc}{subsection}{Final Integration}

In this step, we must integrate the generation terms, the result of the
above, with the Source term, defined the same for holes and electrons
as: \[
S_p = S_n = \frac{P_0\,\eta}{h\nu}
\exp\!\Bigl[-\frac{(x - x_0)^2}{\sigma_x^2}\Bigr]
\exp\!\Bigl[-\frac{(t - t_0)^2}{\sigma_t^2}\Bigr]
\]

In fact, the delta function results in the exponents in the Source
dropping out. This will be seen shortly.

From the above, we can reorganize the generation terms as: \[
G_n(x,t)
= e^{-t/\tau_n}
\sqrt{\frac{1}{4\pi\,D_n\,t}}
\exp\!\Bigl[-\,\frac{\bigl(x - x_0 - \nu_n\,t\bigr)^2}{4\,D_n\,t}\Bigr],
\]

\[
G_p(x,t)
= e^{-t/\tau_p}
\sqrt{\frac{1}{4\pi\,D_p\,t}}
\exp\!\Bigl[-\,\frac{\bigl(x - x_0 + \nu_p\,t\bigr)^2}{4\,D_p\,t}\Bigr]
\]

The relevant delta function property is: \[
\int_{-\infty}^{\infty} f(x)\,\delta(x - a)\,dx = f(a)
\]

Starting with the integration of \(G_n\), which results in our
expression for electron concentration as a function of space and time
\(n(x,t)\):

\[
n(x,t) \;=\; \int G{n}(x,t)\,S_{n}\,dx \;=\;\dots
\]

Within the integral, we have:

\[
\begin{multlined}
G_n(x,t)\,S_n(x,t)
= e^{-t/\tau_n}\,\sqrt{\frac{1}{4\pi\,D_n\,t}}\;\frac{P_0\,\eta}{h\nu}
\;
\\
\exp\!\Bigl[-\frac{(x - x_0 - \nu_n\,t)^2}{4\,D_n\,t}\Bigr]
\;
\\
\exp\!\Bigl[-\frac{(x - x_0)^2}{\sigma_x^2}\Bigr]
\;\exp\!\Bigl[-\frac{(t - t_0)^2}{\sigma_t^2}\Bigr]
\end{multlined}
\]

However, latter two exponents drop out, because of the delta function
property; if one inputs \(x_0\) and \(t_0\), these terms both become
\(exp(0)\) which goes to \(1\).

Therefore, the product of \(G_n(x,t)\) and \(S_n(x,t)\) becomes: \[
G_n(x,t)\,S_n(x,t)
= e^{-t/\tau_n}\,\sqrt{\frac{1}{4\pi\,D_n\,t}}\;\frac{P_0\,\eta}{h\nu}
\;\exp\!\Bigl[-\frac{(x - x_0 - \nu_n\,t)^2}{4\,D_n\,t}\Bigr]
\]

The overall integral becomes:

\[
\begin{multlined}
n(x,t)
= \int G_n(x,t)\,S_n(x,t)\,dx
\\
= \int
e^{-t/\tau_n}\,\sqrt{\frac{1}{4\pi\,D_n\,t}}\;\frac{P_0\,\eta}{h\nu}
\;\exp\!\Bigl[-\frac{((x - x_0) - \nu_n\,t)^2}{4\,D_n\,t}\Bigr]
\,dx
\end{multlined}
\]

Define: \[
\alpha \;=\; e^{-t/\tau_n}\,
\sqrt{\frac{1}{4\pi\,D_n\,t}}\;\frac{P_0\,\eta}{h\nu}\,
\]

Then the carrier density can be written as: \[
\begin{multlined}
n(x,t)
= \int_{x}^{\infty}G_n(x,t)\,S_n(x)\,dx
\\
= \alpha
\int_{x}^{\infty}
\exp\!\Bigl[-\frac{((x - x_0) - \nu_n t)^2}{4\,D_n\,t}\Bigr]
\,dx\,
\end{multlined}
\]

Where \(x\) and \(\infty\) are defined as the bounds of a carrier;
i.e.~one can calculate the concentration from \(x_0\) up to infinitely
far from the \(x_0\) location.

With this result, we now apply the power of a power rule, yielding:

Then the carrier density can be written as \[
= \alpha
\int_{x}^{\infty}
\exp\!\Bigl[-\frac{2((x - x_0)- \nu_n t)}{4\,D_n\,t}\Bigr]
\,dx\,
\]

We can factor out the \(\nu_n\ t\) term, finally giving:

\[
= \alpha \exp(\frac{2\nu t}{4D_{n}t})
\int_{x}^{\infty}
\exp\!\Bigl[-\frac{2(x - x_0)}{4\,D_n\,t}\Bigr]
\,dx\,
\]

This integral can be solved with u-substitution. Where \[
u = \frac{-2(x-x_{0})}{4D_{n}t}
\] and \[
du = -\frac{2}{4D_{n}t}dx
\]

And bounds are appropriately shifted, and defining constants,
\textbf{again with different values than the previous definitions}:

\[
u(\infty) \;=\; -\frac{2(\infty - x_0)}{4D_n t}
\;\Longrightarrow\;
-\infty \;=\; B
\]

\[
u(x) \;=\; -\frac{2(x - x_0)}{4D_n t}
\;=\; A
\]

\[
= \alpha \,
  \exp\!\Bigl(\frac{2\,\nu_n\,t}{4\,D_n\,t}\Bigr)
  \int_{A}^{B} e^{\,u}\,du
\]

\section{Results}\label{results}

\subsection{\texorpdfstring{Carrier Concentrations \(n(x,t)\) and
\(p(x,t)\)}{Carrier Concentrations n(x,t) and p(x,t)}}\label{carrier-concentrations-nxt-and-pxt}

Evaluating this integral and applying the power of a power rule again,
we arrive at our final answer.

\[
n(x,t)
= \frac{P_{0}\,\eta}{h\nu}
  \;e^{-t/\tau_{n}}
  \;\sqrt{\frac{1}{4\pi\,D_{n}\,t}}
  \;\exp\!\Bigl[-\frac{\bigl((x - x_{0}) + \nu_{n}\,t\bigr)^{2}}{4\,D_{n}\,t}\Bigr]
\]

Similarly for holes:

\[
p(x,t)
= \frac{P_{0}\,\eta}{h\nu}
  \;e^{-t/\tau_{p}}
  \;\sqrt{\frac{1}{4\pi\,D_{p}\,t}}
  \;\exp\!\Bigl[-\frac{\bigl((x - x_{0}) - \nu_{p}\,t\bigr)^{2}}{4\,D_{p}\,t}\Bigr]
\]

\subsection{Partial Derivatives of (n(x,t)) and
(p(x,t))}\label{partial-derivatives-of-nxt-and-pxt}

After a \textbf{trivial} derivation involving the chain rule in (x), we
find:

\[
\frac{\partial n(x,t)}{\partial x}
= -\frac{P_{0}\,\eta}{h\nu}
  \;e^{-t/\tau_{n}}
  \;\sqrt{\frac{1}{4\pi\,D_{n}\,t}}
  \;\frac{x - x_{0} + \nu_{n}\,t}{2\,D_{n}\,t}
  \;\exp\!\Bigl[-\frac{\bigl((x - x_{0}) + \nu_{n}\,t\bigr)^{2}}{4\,D_{n}\,t}\Bigr]
\]

Similarly for holes,

\[
\frac{\partial p(x,t)}{\partial x}
= -\frac{P_{0}\,\eta}{h\nu}
  \;e^{-t/\tau_{p}}
  \;\sqrt{\frac{1}{4\pi\,D_{p}\,t}}
  \;\frac{x - x_{0} - \nu_{p}\,t}{2\,D_{p}\,t}
  \;\exp\!\Bigl[-\frac{\bigl((x - x_{0}) - \nu_{p}\,t\bigr)^{2}}{4\,D_{p}\,t}\Bigr]
\]

We developed a Python-based web-app that plots these carrier
concentrations and current, as dependent on their concentrations. The
user can define many relevant parameters such as carrier mobilities or
the strength of an applied electric field and observe these parameters'
influence on carrier propagation and current. The app may be found at
\url{https://pat-trinity-ralph-carrier-propagation-devicephysics-unm.streamlit.app/}.

\section{Conclusion}\label{conclusion}

We derived expressions for charge carriers concentrations, beginning
from Green's functions and applying Fourier and Laplace transforms. The
derived equations are used in a web app modeling these phenomena. The
web app is interactive and visualizes the effect of many parameters on
carrier concentrations and current. Overall, we find that although
lengthy, the derivation was highly informative; and that the model and
app developed are highly extensible and informative in demonstrating the
physics discussed in this class.


% Can use something like this to put references on a page
% by themselves when using endfloat and the captionsoff option.
\ifCLASSOPTIONcaptionsoff
  \newpage
\fi

% trigger a \newpage just before the given reference
% number - used to balance the columns on the last page
% adjust value as needed - may need to be readjusted if
% the document is modified later
%\IEEEtriggeratref{8}
% The "triggered" command can be changed if desired:
%\IEEEtriggercmd{\enlargethispage{-5in}}

% Uncomment when use biblatex with style=ieee
%\renewcommand{\bibfont}{\footnotesize} % for IEEE bibfont size

\pagebreak[3]
% that's all folks
\end{document}

